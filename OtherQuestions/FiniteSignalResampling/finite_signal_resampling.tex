\documentclass[11pt]{article}
\RequirePackage{amssymb, amsfonts, amsmath, latexsym, verbatim, xspace, setspace}
\RequirePackage{tikz, pgflibraryplotmarks}

% By default LaTeX uses large margins.  This doesn't work well on exams; problems
% end up in the "middle" of the page, reducing the amount of space for students
% to work on them.
\usepackage[margin=1in]{geometry}
\usepackage{pgfplots}
\usepackage{hyperref} % FOr URLs
\usepackage[numbers]{natbib} % For citations?  [numbers] needed to avoid Bibliography not compatible with author year citation


\newcommand{\bc}{\mathbf{c}}
\newcommand{\bs}{\mathbf{s}}
\newcommand{\bx}{\mathbf{x}}
\newcommand{\bX}{\mathbf{X}}
\newcommand{\bH}{\mathbf{H}}

\begin{document}
\section{Nyquist for finite signals}

Suppose we have a finite duration signal:
\begin{align}
\bX^\top = \left [x[0], x[1], \ldots x[T-1] \right]
\end{align}
and we wish to decimate it as illustrated in Figure~\ref{fig:decimation}.

% \bibliographystyle{ieeetr}
% \bibliography{./frequency.bib}

\begin{figure}[h]
\begin{center}
\begin{tikzpicture}
  \draw[fill=white, draw=white] (-0.5, 0.7) rectangle (8, -1.7); 
  
  \draw (0,-0.3) node {$x$} ;
  \draw (3.8,-0.3) node {$y$} ;
  \draw (7.6,-0.3) node {$z$} ;

% Input  
    \draw [-latex] (0,-0.5) -- (1.4, -0.5); 

% connection
    \draw [-latex] (3,-0.5) -- (4.4, -0.5); 

% output
    \draw [-latex] (6,-0.5) -- (7.4, -0.5); 

% downsample
    \draw [-latex] (4.9,-0.3) -- (4.9, -0.7); 

  % blocks
  \draw(1.4, 0.3) rectangle (3, -1.3); 
  \draw(4.4, 0.3) rectangle (6, -1.3); 
  \draw(2.2, -0.5) node[text centered, text width=2cm] {\tt LPF$_D$};
  \draw(5.2, -0.5) node[text centered, text width=2cm] {$M$};

  \end{tikzpicture}
  
 \caption{The decimation process.}\label{fig:decimation}
 \end{center}
  \end{figure}
  
  To do this and avoid aliasing, the low-pass filter's ({\tt LPF}$_D$) stop-band must start at $\pi/M$. Assuming this filter is an FIR filter of length $L$ with impulse response $h_{{\tt LPF}_D}$ then we get
  \begin{align}
  y[t] = h_{{\tt LPF}_D}[t] \star x[t]
  \end{align}
  where $\star$ indicates linear convolution.  This makes $y$ of length $T + L - 1$.
  
  The decimator then creates
  \begin{align}
  z[t] = y[tM]
  \end{align}
  so that $z$ will be of length $\frac{T + L - 1}{M}$.
\end{document}

